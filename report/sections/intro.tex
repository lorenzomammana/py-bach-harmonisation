\chapter{Introduzione}\label{ch:intro}

Almeno una volta nella vita, durante la fase di studi, un musicista dovrebbe avere eseguito l'armonizzazione di un corale. Questo compito consiste nel, data una certa melodia, creare tre melodie sottostanti che suonate assieme siano piacevoli all'ascolto. \\
Ciò viene richiesto ai musicisti in quanto è abbastanza aperto per valutare le capacità dello studente, ma allo stesso tempo data la presenza di strutture piuttosto rigorose, è vincolato abbastanza da non richiedere una composizione libera. \\
L'obiettivo del progetto è quello di verificare se questo compito è fattibile tramite algoritmi di apprendimento automatico e se i risultati prodotti siano o meno confondibili con risultati reali.\\
Numerose tecniche sono state presentate in letterature per svolgere questo compito, alcuni approcci basati su reti neurali, altri basati su algoritmi genetici, ma la maggior parte basati su Hidden Markov Models (HMM). \\
Il nostro lavoro è completamente basato sul paper [citare paper] che descrive, oltre ad un HMM in grado di eseguire il task di armonizzazione, anche un secondo HMM in grado di eseguire il compito di ornamentazione, ovvero l'aggiunta di note ininfluenti per la melodia, ma in grado di aggiungere quella fantasia compositiva in più che ci si aspetta da un musicista reale.
