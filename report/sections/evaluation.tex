\chapter{Valutazione risultati}\label{ch:evaluation}
Non avendo conoscenze teoriche in merito alla musica la valutazione dei risultati è stata effettuata semplicemente andando a valutare la gradevolezza uditiva delle melodie composte dal modello.
In particolare sembra essere molto buona l'armonizzazione prodotta dal primo modello di Markov sui vari corali testati, questo soprattutto perchè il compito non è particolarmente complicato e molto spesso suonare più tasti del pianoforte assieme per lo stesso tempo produce suoni gradevoli.\\
I file prodotti tramite sampling basato su melodie note produce risultati simili a quelli di Viterbi, ma effettivamente peggiori all'ascolto.\\
Anche il test eseguito sulla musica moderna produce risultati interessanti, ovviamente la musica prodotta difficilmente potrebbe essere utilizzata realmente, ma soprattutto su parti lente e più vicine ai ritmi di un corale il modello armonizza bene e i risultati sono piuttosto gradevoli all'ascolto.\\
Il modello che esegue l'ornamentazione non ci convince invece particolarmente, molto spesso l'armonizzazione prodotta non è particolarmente gradevole, soprattutto se paragonata ai risultati prodotti dal modello precedente, in particolare abbiamo riscontrato una non riproducibilità dei risultati mostrati dall'autore dell'articolo. I suoi risultati sono eccezionali all'ascolto e sembrano veramente troppo perfetti per essere stati prodotti dal modello, è possibile che vi sia stata una manipolazione umana per rendere più interessante la proposta dell'articolo che, in ogni caso, presenta comunque un ottimo modello!
\chapter{Conclusioni}\label{ch:conclusions}
L'obiettivo ultimo del nostro lavoro era quello di rendere disponibile un modello di Markov per l'armonizzazione di corali funzionante e che non richiedesse di operare su file scritti in PERL, linguaggio molto popolare fino agli anni 2000, ma ormai in lento declino.\\
L'obiettivo è stato pienamente raggiunto, l'intero codice prodotto è scritto in Python ed è perfettamente utilizzabile come base per estendere il modello proposto con ulteriori idee e miglioramenti.\\
Il modello proposto da Allan si è rivelato funzionante, anche se in maniera inferiore a quanto mostrato dall'autore negli esempi che ha reso disponibili pubblicamente; il modello presenta però numerose lacune dovute soprattutto alla rigidità dei file di ingresso e alla scelta di non integrare alcuna informazione riguardante la durata delle note negli stati del modello. Questo causa l'impossibilità di generare automaticamente corali nello stile di Bach, da un certo punto di vista ciò è corretto, non è questo lo scopo che si è prefisso l'autore, da un altro punto di vista sarebbe stato interessante effettuare anche questo compito.\\
Il nostro maggior contributo oltre la traduzione è quello però di aver costruito script particolari in grado di prendere una linea melodica e costruirne l'armonizzazione, questo permette di verificare cosa succederebbe se Bach entrasse a far parte di un certo gruppo musicale. Questo ha fondamentalmente valenza ludica, il che non ne sminuisce comunque il valore, se consideriamo che il settore in questione è uno dei più importanti in ambito informatico.